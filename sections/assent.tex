\section{Assent}

\subsection{Subjectivity and Objectivity}

\begin{enumerate}
    \item % TODO introductory note 368-69
\end{enumerate}

\subsubsection{Restatement First \S\ 227}

\begin{enumerate}
    \item % TODO 369
\end{enumerate}

\subsubsection{\emph{Lucy v. Zehmer}}

\begin{enumerate}
    \item % TODO 370
\end{enumerate}

\subsubsection{\emph{Keller v. Holderman}}

\begin{enumerate}
    \item % TODO 374
\end{enumerate}

\subsubsection{\emph{Raffles v. Wichelhaus}}

\begin{enumerate}
    \item % TODO 374
\end{enumerate}

\subsubsection{Simpson, ``Contracts for Cotton to Arrive: The Case of the Two 
Ships \emph{Peerless}}

\begin{enumerate}
    \item % TODO 375
\end{enumerate}

\subsubsection{\emph{Frigaliment Importing Co. v. B.N.S. Intern. Sales Co.}}

\begin{enumerate}
    \item % TODO 376
\end{enumerate}

\subsubsection{\emph{Oswald v. Allen}}

\begin{enumerate}
    \item % TODO 380
\end{enumerate}

\subsubsection{\emph{Falck v. Williams}}

\begin{enumerate}
    \item % TODO 380
\end{enumerate}

\subsubsection{\emph{Colfax Envelope Corp. v. Local No. 458-3M}}

\begin{enumerate}
    \item % TODO 381
\end{enumerate}

\subsubsection{Intent: \emph{Embry v. Hargadine, McKittrick Dry Goods Co.}}

Contracts generally require a meeting of the minds, but intent is irrelevant 
if the other party could not reasonably know the other's intent.

\begin{enumerate}
    \item Embry supported the company's sales team. After his one-year 
    contract was up, Embry tried to get McKittrick to agree to another 
    contract. After a brief conversation in McKittrick's office where 
    McKittrick said, ``Go ahead, you're all right. Get your men 
    out~.~.~.~''\footnote{Casebook p. 382.} Embry understood his words as 
    assenting to a new one-year contract. McKittrick didn't.
    \item The key issue was whether both parties intended to create a 
    contract.
    \item Contracts generally require a meeting of the minds, but intent is 
    irrelevant if the other party could not reasonably know the other's 
    intent.
    \item Held for Embry.
\end{enumerate}

\subsubsection{Restatement Second \S\S\ 20, 201}

\begin{enumerate}
    \item % TODO supp
\end{enumerate}

\subsubsection{CISG Art. 8}

\begin{enumerate}
    \item % TODO supp
\end{enumerate}

\subsubsection{UNIDROIT Arts. 4.1, 4.2, 4.3}

\begin{enumerate}
    \item % TODO supp
\end{enumerate}

\subsubsection{Principles of European Contract Law Arts. 5.101, 5.102}

\begin{enumerate}
    \item % TODO supp
\end{enumerate}

\subsubsection{CISG and Intent: \emph{MCC-Marble Ceramic Center, Inc. v. 
Ceramica Nuova D'Agostino}}

CISG allows the court to discern the parties' subjective intent, even absent 
outward manifestations.

Would this case have come out differently if US law controlled?

\begin{enumerate}
    \item MCC contracted to buy tiles from D'Agostino. They signed a form 
    contract.
    \item MCC brought suit for D'Agostino's failure to fill orders. D'Agostino 
    argued that it was not obligated to fill the orders because MCC had 
    defaulted on earlier payments.
    \item The form contract gave D'Agostino the right to cancel if MCC failed 
    to pay. MCC argued that it failed to pay because some of the tiles were 
    unsatisfactory---but the contract also had a clause requiring written 
    notice of complaints, which MCC did not give.
    \item MCC argued that it had an oral agreement with D'Agostino that the 
    clauses in the form contract would not apply.
    \item CISG allows inquiry into subjective intent, ``even if the parties 
    did not engage in any objectively ascertainable means of registering this 
    intent.''\footnote{Casebook p. 387.}
    \item Three affiants testified to MCC's intent to nullify the clauses. 
    While the affidavits may have been conclusory, they at least presented a 
    triable issue of fact, so summary judgment was inappropriate.
\end{enumerate}

\subsubsection{\emph{Mayol v. Weiner Companies, Ltd.}}

\begin{enumerate}
    \item Mayol contracted to buy a piece of property which Weiner sold on 
    behalf of the owner. The contract included a clause granting possession 
    ``on or before November 1, 1979 subject to tenant's rights.'' Mayol paid a 
    \$1,000 deposit.
    \item It turned out that the tenant had a right to purchase. Upon learning 
    this, Mayol breached and sued to recover his deposit.
    \item Mayol had asked about the lease but Weiner had not told him about 
    the tenant's purchase option until after the sale contract was complete. 
    The court held for Mayol, reasoning that he had no reason to believe that 
    he was buying property subject to a purchase option, and the seller had no 
    reason to think that Mayol believed so, either.
\end{enumerate}

\subsubsection{Objective and Subjective Elements in Interpretation}

\begin{enumerate}
    \item Classical contract law largely disregarded the parties' intent.
    \item There are four principles of interpretation in modern contract 
    law:\footnote{Casebook p. 394--95.}
    \begin{enumerate}
        \item The more reasonable meaning prevails If both parties attach 
        different subjective meanings to an expression  and they 
        are not equally reasonable.
        \item But if the two meanings are equally reasonable, neither 
        prevails.
        \item If the parties attach the same meaning, that meaning prevails 
        even if it is unreasonable.
        \item If A and B attach different meanings, and A knows B's meaning 
        but B doesn't know A's, B's meaning prevails even if it is less 
        reasonable.
    \end{enumerate}
\end{enumerate}

\subsubsection{\emph{Berke Moore Co. v. Phoenix Bridge Co.}}

\begin{enumerate}
    \item Mutual understanding is not private and is therefore valid.
\end{enumerate}

\subsection{Problems of Interpreting Purposive Language}

% TODO 397-406

\subsection{Usage, Course of Dealing, and Course of Performance}

% TODO 406-314

