\section{Mistake and Unexpected Circumstances}

\subsection{Mistake}

\subsubsection{Unilateral Mistakes (Mechanical Errors)}

\paragraph{The Nolan Ryan Baseball Card Case}

\begin{enumerate}
    \item Bryan Wrzesinski bought a baseball card worth \$1,200 for \$12. The 
    cashier apparently didn't know its worth. The owner of the store sued to 
    recover either the card or damages.\footnote{Casebook p. 727.}
    \item The parties settled, auctioning of the card and giving the proceeds 
    to charity.
    \item The facts are ambiguous, but under any scenario, Bryan either misled 
    the cashier or knew that the price tag was the result of a mechanical 
    error.
\end{enumerate}

\paragraph{Restatement Second \S\S\ 154, 154}

\begin{enumerate}
    \item % TODO supp
\end{enumerate}

\paragraph{Unilateral Mistake}

\begin{enumerate}
    \item ``~.~.~.~relief will not be granted unless the other party has 
    either not relied \emph{or} cannot be restored to his precontractual 
    position by the award of reliance damages.''\footnote{Casebook p. 728.}
\end{enumerate}

\subsubsection{Mistakes in Transcription; Reformation}

\paragraph{\emph{Travelers Ins. Co. v. Bailey}}

\begin{enumerate}
    \item Bailey bought life insurance. The plan he applied for was for 
    \$5,000 plus a retirement annuity for \$500 per \emph{year}.
    \item His policy documents said that the annuity was \$500 per 
    \emph{month}.
    \item Bailey brought the issue to Travelers' attention, and it issued a 
    new policy.
    \item ``Where, as here, an antecedent contract has been established by the 
    requisite measure of proof, equity will act to bring the erroneous writing 
    into conformity with the true agreement.''\footnote{Casebook pp. 729--30.} 
    Unilateral mistakes support reformation.
    \item ``~.~.~.~we hold that where there has been established beyond a 
    reasonable doubt a specific contractual agreement between parties, and a 
    aubsequent erroneous rendition of the terms of the agreement in a material 
    particular, the party penalized by the error is entitled to reformation, 
    if there has been no prejudicial change of position by the other party 
    while ignorant of the mistake.''\footnote{Casebook pp. 730--31.}
\end{enumerate}

\subsection{The Effect of Unexpected Circumstances}

% TODO 765 ff.
