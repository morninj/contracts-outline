\section{Form Contracts in Commercial and Consumer Settings}

% TODO 639-641

\subsection{Contract Formation in a Form-Contract Setting}

\subsubsection{Battle of the Forms: The Basic Issues}

% TODO 641-657

\subsubsection{``Rolling Contracts''}

\paragraph{Shrinkwrap Agreements: \emph{ProCD, Inc. v. Zeidenberg}}

\begin{enumerate}
    \item ProCD sold consumer information databases on CDs. It sold at a 
    higher price to commercial users. It used a license, encoded on the CDs 
    and printed in the user manual, to enforce its price discrimination 
    system. The user was required to agree to the license agreement each time 
    the application ran.
    \item Zeidenberg published ProCD's databases online. ProCD sued for an 
    injunction on the ground that Zeidenberg had violated the license 
    agreement.
    \item The district court held that the licenses were ineffectual because 
    their terms did not appear on the outside of the packages.
    \item Zeidenberg that he was only bound by terms that were printed on the 
    outside of the box.
    \item The Seventh Circuit here held that there are plenty of examples of 
    [t]ransactions in which the exchange of money precedes the communication 
    of detailed terms---e.g., insurance, airline tickets, consumer 
    electronics, or software bought online. The customer can assent to the 
    additional terms by using the product, or he can reject the terms by using 
    the product.
    \item The UCC permits this type of ``money now, terms later'' agreement.
    \item ``In the end, the terms of the license are conceptually identical to 
    the contents of the package.''\footnote{Casebook p. 669.}
    \item Reversed.
\end{enumerate}

\paragraph{Applying \emph{ProCD} to Hardware: \emph{Hill v. Gateway 2000}}

\begin{enumerate}
    \item The hills bought a Gateway PC. Inside the box was an agreement which 
    included an arbitration clause. The agreement indicated that it was 
    binding if the customer did not return the computer within 30 days. The 
    Hills kept the computer for more than 30 days.
    \item The Hills sued to invalidate the arbitration clause. The district 
    court refused to enforce the clause.
    \item There was no reason to limit \emph{ProCD} to software. For instance, 
    why require a salesperson to read an entire four-page document over the 
    phone to a customer?
    \item The Hills argued that \emph{ProCD} should only apply to executory 
    contracts, and did not apply in this case because performance was complete 
    when the Hills opened the box. The court held that this argument was wrong 
    legally because the issue is contract formation, not performance, and 
    wrong factually because both contracts were incompletely performed (e.g., 
    the Hills invoked Gateway's warranty, which was available to them only 
    after they opened the box and read the additional terms).
    \item It didn't matter whether there were terms on the outside of the 
    Gateway box.
    \item The agreement, including the arbitration clause, was enforceable.
\end{enumerate}

\paragraph{Clickwrap Agreements: \emph{Specht v. Netscape Communications 
Corp.}}
~\\\\
Clickwrap agreements require ``[r]easonably conspicuous notice of the 
existence of contract terms and unambiguous manifestation of assent to those 
terms by consumers~.~.~.''\footnote{Casebook p. 686.}

\begin{enumerate}
    \item Plaintiffs in three related class actions downloaded Netscape 
    SmartDownload. They alleged that SmartDownload contained code to allow 
    Netscape to eavesdrop on their downloading activity.
    \item Users who downloaded SmartDownload did not have to agree to a 
    clickwrap agreement to download or install the software. The license terms 
    were on the download page, but the use would have had to scroll down to 
    find them.
    \item One of the terms in the SmartDownload license agreement was an 
    arbitration clause.
    \item In district court, Netscape moved to compel arbitration. The court 
    held that the page did not alert users to the license agreement nor 
    require their assent.
    \item Judge Sotomayor (Second Circuit):
    \begin{enumerate}
        \item Offerees are not bound by inconspicuous terms of which he is 
        unaware in a document that is not obviously a 
        contract.\footnote{Casebook p. 682.}
        \item ``Inquiry notice'': circumstances to put a reasonable person on 
        actual notice. Netscape argued that users were on inquiry notice of 
        the license agreement. But the court held that the terms were not 
        apparent to the user nor obviously contractual.
        \item ``We conclude that in circumstances such as these, where 
        consumers are urged to download free software at the immediate click 
        of a button, a reference to the existence of license terms on a 
        submerged screen is not sufficient to place consumers on inquiry or 
        constructive notice of those terms.''\footnote{Casebook p. 684.}
        \item Affirmed.
    \end{enumerate}
\end{enumerate}

\subsection{Interpretation and Unconscionability in a Form Contract Setting}

\subsubsection{Reformation and Mutual Mistake: \emph{Sardo v. Fidelity \& 
Deposit Co.}}

\begin{enumerate}
    \item Sardo bought insurance to insure the jewelry in his store against 
    theft. Fidelity issued a policy covering ``money and securities.'' Sardo 
    assumed that ``securities'' included jewelry, but the contract included a 
    specific definition of ``securities'' that clearly did not include 
    jewelry.
    \item Sardo's store was robbed. Sardo sued to force Fidelity to reimburse 
    him for the lost jewelry.
    \item The trial court held that the contract should be reformed to change 
    ``securities'' to ``jewelry.''\footnote{Casebook p. 693.}
    \item The appellate court here held that reformation is appropriate only 
    in the case of mutual mistake. But here, only Sardo made a mistake. 
    Reversed.
\end{enumerate}

\subsubsection{Llewellyn, ``The Common Law Tradition: Deciding Appeals''}

\begin{enumerate}
    \item Form contracts are attractive. But they can take on ``a massive and 
    almost terrifying jug-handled character.''\footnote{Casebook p. 694.} 
    Nobody really reads them.
    \item So, nobody really assents to boilerplate terms. Rather, people only 
    really assent to the dickered terms.
    \item ``[A]ny contract with boiler-plate results in \emph{two} several 
    contracts: the \emph{dickered} deal, and the collateral one of 
    \emph{supplementary} boiler-plate.''\footnote{Casebook p. 696.}
\end{enumerate}

\subsubsection{\emph{Weaver v. American Oil Co.}}

\begin{enumerate}
    \item American Oil leased a gas station to Weaver each year. The lease 
    included a clause that exculpated American Oil from liability for its 
    negligence and compelled Weaver to indemnify them for damages resulting 
    from its negligence.
    \item Weaver hadn't finished high school. He never read the lease and 
    nobody explained its terms to him. He lacked any awareness of the 
    indemnity clause.
    \item If the contract had been for the sale of goods, it would have been 
    unconscionable under UCC \S\ 2-302.\footnote{Casebook p. 697.}
    \item The clause was buried in the contract.
    \item The parties' actual understanding of the agreement should outweigh 
    the parol evidence rule. This agreement ``should not be enforceable on the 
    grounds that the provision is contrary to public 
    policy.''\footnote{Casebook p. 699.}
    \item Indemnity clauses are fine if the parties agree to them knowingly 
    and willingly.
\end{enumerate}

