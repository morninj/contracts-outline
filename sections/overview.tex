\section{Overview}

\subsection{Classical and Modern Contract Law}

\begin{enumerate}
    \item Contract law reasoning:\footnote{Casebook p. 83 
    ff.}
    \begin{enumerate}
        \item \emph{Substantive legal reasoning}: validity of a doctrine turns 
        on normative considerations (morals, policy, etc.).
        \item \emph{Formal legal reasoning}: law consists of doctrines that 
        are autonomous from policy, morality, and experience.
        \item \emph{Axiomatic legal reasoning}: ``fundamental doctrines can be 
        established on the ground that they are 
        self-evident.''\footnote{Casebook p. 83.}
        \item \emph{Deductive legal reasoning}: most doctrines follow from 
        syllogisms beginning with more fundamental doctrines.
        \item Classical contract law coupled axiomatic and deductive 
        reasoning.
        \item By contrast, modern contract law reasoning justifies doctrines 
        on the basis of morality, policy, and experience.
    \end{enumerate}
    \item Contract law can be plotted along four axes:
    \begin{enumerate}
        \item Objectivity (directly observable state of the world) 
        $\leftrightarrow$ subjectivity (mental state).
        \item Standardization (depends on abstract variables) 
        $\leftrightarrow$ individualization (depends on situation-specific 
        variables).
        \item Static (depends on what occurred at the moment the contract was 
        formed) $\leftrightarrow$ dynamic (depends on moving streams of events 
        before and after the contract).
        \item Binary (e.g., either no damages or full expectation damages) 
        $\leftrightarrow$ multifaceted (e.g., no damages, expectation damages, 
        reliance damages, restitution damages).
    \end{enumerate}
    \item Many other limits on contracts are surrogates for 
    unconscionability.\footnote{Casebook p. 86.}
\end{enumerate}

\subsection{Consideration}

\subsubsection{Donative Promises}

\begin{enumerate}
    \item Donative promises are not enforceable. \emph{Dougherty v. Salt}.
    \item But, gifts cannot be revoked once given.
    \item Acceptance alone is not consideration, but at some point, an act of 
    acceptance involves so much effort that it becomes consideration (the 
    ``tramp example'').
    \item Courts generally do not assess the adequacy of consideration.
    \item Two conceptions:
    \begin{enumerate}
        \item \emph{Broad}: ``consideration'' refers collectively to the 
        things that make contracts legally enforceable---e.g., bargain or 
        reliance.\footnote{Casebook p. 8.}
        \item \emph{Narrow}: ``consideration'' is the same thing as 
        ``bargain.'' The Restatement Second adopts this approach, known as the 
        \textbf{bargain theory of consideration}.
    \end{enumerate}
    \item \textbf{Bargain promise}: the condition is the price---e.g., I'll 
    give you \$20 if you mow my lawn.
    \item \textbf{Conditional donative promise}: the condition is the means 
    to make the gift---e.g., I'll buy you a car if you pick one that costs 
    less than \$15,000.
\end{enumerate}

\subsubsection{Form}

\begin{enumerate}
    \item Historically, formal elements (wax seals, etc.) indicated the 
    validity of an agreement. Today, form has little effect.
    \item Ritual is not enough. To be valid, consideration must impose a legal 
    obligation. \emph{Schnell v. Nell}.
\end{enumerate}

\subsubsection{Reliance and Promissory Estoppel}

\begin{enumerate}
    \item Early courts did not recognize reliance as grounds for enforcing an 
    agreement. \emph{Kirksey v. Kirksey} (``Dear Sister Antillico~.~.~.~'').
    \item \textbf{Estoppel in pais} (or equitable estoppel): if A made a 
    statement and B foreseeably relied on that statement, A is estopped from 
    denying that statement's truth. Based on a \emph{statement of fact}.
    \item \textbf{Promissory estoppel}: a promise is binding if the promisee 
    reasonably relied on it. Based on a \emph{promise}. \emph{Feinberg v. 
    Pfeiffer}. \emph{Cf.} \emph{Hayes v. Plantations Steel Co.}
    \begin{enumerate}
        \item Restatement \S\ 90: ``A promise which the promisor should 
        reasonably expect to induce action or forbearance on the part of the 
        promisee or a third person and which does induce such action or 
        forbearance is binding if injustice can be avoided only by enforcement 
        of the promise. The remedy granted for breach \textbf{may be limited 
        as justice requires}.''
    \end{enumerate}
    \item Reliance damages can compensate a party for injuries arising from 
    justified reliance. \emph{D \& G Stout, Inc. v. Bacardi Imports, Inc.}
    \item Courts are hesitant to award expectation damages for lost profits, 
    but they will if the venture was reasonably likely to be profitable. 
    \emph{Walters v. Marathon Oil Co.}
\end{enumerate}

\subsubsection{The Bargain Principle}

\begin{enumerate}
    \item Consideration arises from a \textbf{bargained-for exchange}. Moral 
    obligations are not enough.
    \item Courts will not evaluate the adequacy of consideration as long as 
    the consideration created a legal obligation.
    \item Consideration can arise from an act or forbearance. \emph{Hamer v. 
    Sidway}, \emph{Davies v. Martel Laboratory Services, Inc.} \emph{See} 
    Restatement \S\S\ 71, 72, 79.
    \item Bad (or great) deals are enforceable if there is consideration. 
    \emph{Hancock Bank \& Trust Co. v. Shell Oil Co.}
    \item Loan agreements are enforceable even if the amount to be repaid is 
    orders of magnitude larger than the amount loaned. \emph{Batsakis v. 
    Demotsis.}
    \item \textbf{Duress} will invalidate a contract. But, driving a hard 
    bargain is not wrongful. \emph{Chouinard v. Chouinard}.
    \item The rules against duress prevent a rescuer from unfairly profiting 
    from the rescue. However, the rules of \textbf{salvage} create an 
    incentive to rescue by rewarding the rescuer and compensating him for his 
    costs, striking a balance between the rescuer's effort and the victim's 
    need. \emph{Post v. Jones}.
    \begin{enumerate}
        \item \emph{The Desperate Traveler}: T is stranded in the desert. G 
        passes by and offers a ride for two-thirds of T's wealth or \$100,000, 
        whichever is more.
        \item Classical contract law would enforce the agreement because the 
        rescuer was not responsible for putting the promisor in a position of 
        distress. Salvage rules offer a solution by preventing the rescuer 
        from exploiting the one in distress while also creating an incentive 
        to rescue.
    \end{enumerate}
    \item Price gouging is wrong.
\end{enumerate}

\subsubsection{Unconscionability}

\begin{enumerate}
    \item Unconscionability arises when there is a \textbf{lack of meaningful 
    choice} and the terms of the agreement are \textbf{unreasonable favorable 
    to one party}. Large disparities in bargaining power can also signal 
    unconscionability.  \emph{Williams v. Walker-Thomas Furniture Co.}
    \item UCC \S\ 2-302: courts can rescind or reform unconscionable 
    contracts. \emph{Accord} Restatement \S\ (invaliding agreements with 
    \emph{gross} disparities in consideration or bargaining power, but holding 
    that imbalances do not always indicate unconscionability).
    \item There's no easy way to explain unconscionability. The UCC and the 
    Restatement struggle to define it. As soon as we define it, people will 
    draft around it. But we know it when we see it.
    \item \textbf{Substantive unconscionability}: one-sidedness.
    \item \textbf{Procedural unconscionability}: fine-print surprises.  
    \emph{Maxwell v. Fidelity Fin. Servs., Inc.}
\end{enumerate}

\subsubsection{Mutuality}

\begin{enumerate}
    \item Under classical contract law, there was no contract without mutual 
    benefit and detriment---i.e., both parties must be bound, or neither is.
    \item ``As a contract defense, the mutuality doctrine has become a 
    faltering rampart to which a litigant retreats at his own 
    peril.''\footnote{Casebook p.  100.} \emph{Helle v. Landmark, Inc.}
    \item Restatement \S\ 79: if there is consideration, there are no 
    requirements of gain/loss, equivalence in value, or ``mutuality of 
    obligation.''
    \item \textbf{Illusory promises}: there is no consideration without 
    commitment. Today, courts will find a lack of mutuality only if one party 
    retains \textbf{complete} discretion to abandon the agreement, which 
    rarely occurs. We also require parties to perform in good faith. Today, 
    courts generally look beyond the contract's literal terms to determine 
    whether the parties intended the agreement to be mutually binding. The 
    intent of the parties outweighs \textbf{``primitive formalism.''} 
    \emph{Wood v. Lucy, Lady Duff-Gordon}.
    \item As long as there is sufficient consideration, unequal terms do not 
    invalidate a contract. \emph{Lindner v. Mid-Continent Petroleum Corp.}
    \item \textbf{Satisfaction clauses} constitute consideration when courts 
    read them as requiring good faith. Otherwise, the satisfaction clause 
    would not involve a commitment, so the promise would be illusory. 
    \emph{Mattei v. Hopper}.
    \item Any bargained-for act or forbearance constitutes consideration. But 
    de minimus non curat lex. \emph{Harris v. Time, Inc.}
    \item Requirements contracts and output contracts:
    \begin{enumerate}
        \item \emph{Requirements contract}: a seller agrees to provide as much 
        of a good as the buyer requires.
        \item \emph{Output contract}: a buyer agrees to buy all of a seller's 
        output.
        \item Traditionally, ``courts often refused to enforce requirements 
        contracts where the buyer could choose to have no 
        requirements.''\footnote{Casebook p. 103.} But both parties shrunk 
        their realm of choice, so there was mutual consideration.
        \item UCC \S\ 2-306 requires good faith in requirements contracts.
        \item ``~.~.~.~a modern court would almost certainly hold that all 
        requirements and output contracts have consideration.''
    \end{enumerate}
\end{enumerate}

\subsubsection{Legal Duty}

\begin{enumerate}
    \item Performance of an existing duty is not consideration. \emph{Slattery 
    v. Wells Fargo Armored Serv. Corp.}, \emph{but see} \emph{Denney v. 
    Reppert}.
    \item Withholding performance does not create consideration for a 
    modification. It looks more like extortion.\emph{Lingenfelder v. 
    Wainwright Brewery Co.}
    \item The preexisting duty rule helps prevent extortion. \emph{Alaska 
    Packers' Association v. Domenico}. On the other hand, where a modification 
    involves consideration, the preexisting duty rule fails to stop extortion. 
    \emph{Austin Instrument, Inc. v. Loral} (where the court found valid 
    consideration, but nonetheless held for Loral on a theory of duress).
    \item The preexisting duty rule can also allow parties to renege on good 
    faith modifications. For example, A hires B to build a house for 
    \$100,000. The cost of materials rises during construction. A and B agree 
    to raise the contract price to \$150,000. When the job is done, A backs 
    out, claiming (1) that there was no consideration for the modification and 
    (2) that B was performing a preexisting duty.
\end{enumerate}

\subsubsection{Modification}

\begin{enumerate}
    \item Today, courts generally do not require consideration for good faith 
    modifications. \emph{Schwartzreich v. Bauman-Basch, Inc.}, \emph{Angel v. 
    Murray}.
    \item \emph{Accord}: an agreement under which a new obligation replaces an 
    existing obligation.
    \item \emph{Executory accord}: an unperformed accord.
    \item \emph{Satisfaction}: performance of the accord.
    \item \textbf{Substituted contract}: the earlier agreement is immediately 
    discharged.
    \item Substituted contract vs. accord:
    \begin{enumerate}
        \item Courts are likely to find that an accord is a substituted 
        contract if the original duty ``was disputed, unliquidated, had not 
        matured, and involved a performance other than the payment of 
        money.''\footnote{Casebook pp. 136--37.} For instance, for 
        agricultural services, A agreed to pay B in cattle. They disagreed 
        over how many cattle were due. They agreed that A would pay three 
        sheep instead. Courts will likely treat the accord as a substituted 
        contract.
        \item Courts are likely to find that an accord is \emph{not} a 
        substituted contract if the original duty ``was undisputed, 
        liquidated, had matured, and involved the payment of 
        money.''\footnote{Casebook p. 137.} For instance, A agreed to pay 
        \$900 for B's services, and A does not dispute B's claim. They agree 
        that A will give two sheep instead of paying \$900.  Courts will 
        likely \emph{not} find that the accord is a substituted contract.
    \end{enumerate}
    \item Restatement \S\ 89: a promise modifying an earlier contract is 
    binding when:
    \begin{enumerate}
        \item It's fair;
        \item Provided by statute; or
        \item ``justice requires enforcement'' if a party relies on it.
    \end{enumerate}
\end{enumerate}

\subsubsection{Waiver}

\begin{enumerate}
    \item Parties can waive contractual terms. \emph{Clark v. West}.
    \item Courts are split on whether waiver requires reliance. \emph{Nassau 
    Trust Co. v. Montrose Concrete Products Co.}
\end{enumerate}

\subsubsection{Past Consideration}

\begin{enumerate}
    \item Traditionally, there are \textbf{three areas} in which a promise to 
    discharge a prior unenforceable obligation is binding:
    \begin{enumerate}
        \item A promise to pay a debt barred by the statute of limitations.
        \item A promise by an adult to pay a debt incurred when the person was 
        underage.
        \item A promise to pay a debt that has been discharged in bankruptcy.
    \end{enumerate}
    \item Moral obligations do not make promises enforceable. Past 
    consideration is required. \emph{Mills v. Wyman}, \emph{Harrington v. 
    Taylor}. But, equity can compel courts to depart from the rule. 
    \emph{Webb v. McGowin}.
    \begin{enumerate}
        \item ALI: past consideration is \emph{sometimes} binding. Some cases 
        are ``gratuitous transactions'' (with no consideration) and 
        ``quasi-contracts'' (with consideration, if justice 
        requires).\footnote{Casebook p. 160--61.}
    \end{enumerate}
    \item If A confers a benefit on B without B's prior request, the 
    subsequent relationship could fall into three categories:
    \begin{enumerate}
        \item B is \textbf{legally obligated} to compensate A under the law of 
        \emph{unjust enrichment}---for instance, if A paid B money by mistake. 
        The Restatement (Second) adopts this approach, though courts 
        traditionally did not recognize it.
        \item B is \textbf{morally but not legally obligated} to compensate 
        A---for instance, if B has suffered a loss on rescuing A. \emph{Mills 
        v. Wyman.}
        \item B is \textbf{neither morally nor legally obligated} to 
        compensate A---e.g., ordinary gifts.
    \end{enumerate}
\end{enumerate}

\subsubsection{The Limits of Contract}

\begin{enumerate}
    \item Agreements do not always constitute a binding agreement---e.g., two 
    people agreeing to go to a party.
    \item Should courts enforce in vitro fertilization agreements? Until what 
    point can parties change their minds? \emph{In Re the Marriage of Witten.} 
    What about surrogacy agreements? \emph{R.R. v. M.H.}
    \item 42 U.S.C. \S\ 247(e): organs cannot be bought.
    \item Posner: any bureaucratic solution will be flawed, so let's allow the 
    sale of organs on the open market.
    \item Radin: ``In precluding sales but not gifts, market-inalienability 
    places some things outside the marketplace but not outside the realm of 
    social intercourse.''\footnote{Casebook p. 186.} Some things shouldn't be 
    commodified---children, organs, sexual services, and so on.
\end{enumerate}

\subsection{Remedies}

\subsubsection{Overview}

\begin{enumerate}
    \item \textbf{Expectation}: puts promisee in the position he would have 
    been in if the contract had been \emph{performed}. You get what you 
    bargained for.
    \item \textbf{Reliance}: puts promisee in the position he would have been 
    in 
    if the contract had \emph{not been made}.
    \item \textbf{Restitution}: restores to the promisee any benefit he 
    conferred to the promisor.
    \item Expectation damages are the difference between promised performance 
    and actual performance---for instance, between a perfect hand and a hairy, 
    scarred hand. \emph{Hawkins v. McGee}.
    \item Damages compensate a party for actual losses. No punitive damages. 
    \emph{U.S. Naval Inst. v.  Charter Commc'ns, Inc.}
    \item Classical contract law did not permit a seller to profit from 
    breach. \emph{Coppola Enters., Inc. v. Alfone}. Some courts viewed it as a 
    breach of the covenant of good faith and fair dealing. \emph{Greer Props., 
    Inc. v. LaSalle Nat'l Bank}.
    \item Posner argues for efficient breach. Eisenberg responds that 
    efficient breach is ultimately not so efficient.
\end{enumerate}

\subsubsection{Expectation Damages}

\begin{enumerate}
    \item \textbf{Breach of contract to perform services}:
    \begin{enumerate}
        \item Two remedies: \textbf{cost of completion} and \textbf{diminution 
        in value}.
        \item If a contractor leaves a job unfinished, the client can recover 
        \textbf{cost of completion} damages. \emph{Louise Caroline Nursing 
        Home, Inc. v. Dix Construction Co.}
        \item Where cost of performance exceeds \textbf{diminution in value}, 
        courts are likely to award damages only for diminution in value.  
        \emph{Peevyhouse v. Garland Coal \& Mining Co.} Courts are wary of 
        awarding cost of completion damages when the cost is \textbf{grossly 
        disproportionate} to the benefits (\emph{H.P. Droher \& Sons v. 
        Toushin}) or when it would cause \textbf{waste} (\emph{Eastern 
        Steamship Lines, Inc. v.  United States}).
        \item Courts are more likely to award cost of completion damages when 
        the defects are aesthetically important (e.g., a fancy swimming 
        pool---\emph{City School Dist. of the City of Elmira v.  McLane Const.  
        Co.}) or the project involves someone's home (\emph{Fox v. Webb}).  
        \item Calculating damages for partially performed services:
        \begin{enumerate}
            \item Contract price - value of performance completed - amount 
            paid by owner prior to breach. Restatement Second \S\ 346.
            \item Example: A contracts to build a house for B for \$100,000. B 
            breaches halfway through. A would have to spend \$60,000 to 
            finish.  A's damages are the contract price (\$100,000) - cost of 
            completion that the contractor saved (\$60,000) = \$40,000. A can 
            recover \$40,000 minus any payments already made.
        \end{enumerate}
    \end{enumerate}
    \item \textbf{Breach of a contract for the sale of goods}:
    \begin{enumerate}
        \item The UCC governs.
        \item \textbf{Seller's breach}:
        \begin{enumerate}
            \item Two remedies: \textbf{specific relief} and \textbf{damages}.
            \item For defective goods, courts might award damages for the 
            \textbf{cost of repair} in excess of the purchase price.  UCC \S\ 
            2-714(2) sets the measure of direct damages for breach of warranty 
            as the difference between the value of the goods as warranted and 
            the value of the goods as accepted, often approximated as the cost 
            of repair.
            \item When the seller fails to deliver goods sold, the buyer has 
            two options:
            \begin{itemize}
                \item \emph{Cover}: the buyer purchases a replacement. Damages 
                are the difference between the replacement price and the 
                contract price. UCC \S\ 7-212 and \emph{KGM Harvesting Co. v. 
                Fresh Network.}
                \item \emph{Hypothetical cover} or \textbf{market price} 
                damages: the buyer recovers the difference between the market 
                price at the time of the breach and the contract price. UCC 
                \S\ 7-213.
            \end{itemize}
            \item When covering, the UCC 2-713 allows goods of a different 
            level of quality. \emph{Egerer v. CSR West, LLC.}
            \item When a buyer covers, can he still recover market price 
            damages under UCC \S\ 7-213, rather than the actual cost of cover 
            under \S\ 2-712?
            \begin{itemize}
                \item Example: a buyer agrees to buy a steamroller for 
                \$1,000. The market price for steamrollers is \$1,300 at the 
                time of the breach. The buyer waits six months to cover, when 
                the market price has dropped to \$500. Under \S\ 2-712, he can 
                recover nothing, but under 2-713, he can recover \$300.
                \item The UCC has been updated to force covering buyers to use 
                \S\ 7-212, but most states have not yet adopted the change.
            \end{itemize}
            \item If replacing the goods is not possible, courts will award 
            the \textbf{cost of repair}, even if it exceeds the purchase 
            price.  \emph{Continental Sand \& Gravel, Inc. v. K \& K Sand \& 
            Gravel, Inc.}
            \item Buyers can recover reasonable \textbf{incidental damages} 
            related to the goods in question. \emph{Delchi Carrier SpA v. 
            Rotorex Corp.}
        \end{enumerate}
        \item \textbf{Buyer's breach}:
        \begin{enumerate}
            \item When buyer breaches and the seller justifiably withholds 
            goods, UCC \S\ 2-718 allows the buyer to win restitution by the 
            amount his payments exceed (1) specified liquidated damages or (2) 
            20\% of the total performance, or \$500, whichever is smaller.
            \begin{enumerate}
                \item However, the buyer's restitution damages can be offset 
                if the seller shows damages under another UCC 
                provision---e.g., \textbf{lost profits} under UCC \S\ 2-708. 
                \emph{Neri v. Retail Marine Corp.}
            \end{enumerate}
            % TODO UCC 2-708, esp 708(2)
            \item A \textbf{lost volume seller} has ``the capacity to supply 
            the breached units in addition to what it actually sold.'' Lost 
            volume sellers should be able to recover lost profits from the 
            loss of a single sale. An exception occurs when the volume would 
            be so high that sales would no longer be profitable (e.g., 
            McDonalds can sell a billion hamburgers, but it could not sell 1.5 
            billion without significant infrastructure expansions---so, the 
            extra 0.5 billion sales would not have been profitable, so 
            McDonalds should not be able to recover for lost sales, even 
            though it is a lost volume seller.) \emph{R.E. Davis Chemical Corp. 
            v. Diasonics, Inc.}
            \begin{enumerate}
                \item Second-hand car dealers are not lost volume sellers 
                because each used car is unique. If a buyer of a used car 
                breaches, the seller cannot recover if it sells the same car 
                to another buyer for the same price. \emph{Lazenby Garages 
                Ltd. v. Wright}.
            \end{enumerate}
        \end{enumerate}
    \end{enumerate}
    \item \textbf{Mitigation and contracts for employment}:
    \begin{enumerate}
        \item Both parties have a \textbf{duty to mitigate} damages after 
        breach. \emph{Rockingham County v. Luten Bridge Co.}
        \item The duty to mitigate applies even if the goods are different 
        than what the contract specified. \emph{Maden v. Murray \& Sons, Inc.}
        \item The defaulting party cannot dictate how the other party must 
        mitigate. As long as the mitigating party acted \textbf{reasonably}, 
        he has fulfilled his duty to mitigate. \emph{In Re Kellet Aircraft 
        Corporation}. Similarly, the mitigating party need not make 
        unreasonable effort to mitigate. \emph{Bank One, Texas N.A. v.  Taylor}.
        \item Where the defendant had an equal opportunity to mitigate loss, 
        he cannot reasonably contend that the plaintiff failed to mitigate. 
        \emph{S.J. Groves \& Sons Co. v. Warner Co.}
        \item % TODO UCC 2-704, 2-715
        \item Employees have a duty to mitigate, but only by seeking 
        \textbf{similar employment}. \emph{Shirley MacLaine Parker v.  
        Twentieth Century-Fox Film Corp.}, \emph{Punkar v. King Plastic 
        Corp.} Employers must compensate wrongfully discharged employees for 
        reasonable expenses incurred in mitigation. \emph{Punkar v. King 
        Plastic Corp.}
        \item Plaintiffs can recover \textbf{reputation damages} if they can 
        prove specific losses. \emph{Redgrave v. Boston Symphony Orchestra, 
        Inc.}\footnote{Squibbed on casebook pp. 276--77.}
        \item English courts have allowed recovery for loss of opportunity to 
        appear before the public, but American courts have not, absent 
        evidence of specific losses.
    \end{enumerate}
    \item \textbf{Foreseeability}:
    \begin{enumerate}
        \item \emph{Hadley v. Baxendale}:
        \begin{enumerate}
            \item \emph{First rule}: the party in breach is liable for 
            reasonably foreseeable damages.
            \item \emph{Second rule}: if the party in breach is aware of 
            special circumstances, he is liable for extra loss those 
            circumstances create. But he is not liable for the extra loss if 
            he was not aware of the special circumstances.
            \item Courts apply \emph{Hadley} to determine whether a 
            \textbf{type} or \textbf{amount} of damages is reasonably 
            foreseeable.
            \item Courts sometimes deviate from the \emph{Hadley} 
            foreseeability rule.\footnote{Casebook pp. 288--89.}
        \end{enumerate}
        \item A commercial machine has only commercial uses. It's reasonably 
        foreseeable that failing to deliver the machine will hinder commerce. 
        \emph{Victoria Laundry (Windsor) Ltd. v. Newman Indus. Ltd.}
        \item Fluctuations in market price are foreseeable. So, for instance, 
        shippers are liable if their delay causes goods to be sold at a lower 
        market price. \emph{Koufos v. C. Czarnikow, Ltd. (The Heron II)}.
        \item If shipments of commercial machinery are delayed, the shipper 
        can be liable for rental charges in the interim, as long as the 
        equipment has an obvious commercial use. \emph{Hector Martinez \& Co. 
        v. Southern Pacific Transp. Co.}
    \end{enumerate}
    \item \textbf{Certainty}:
    \begin{enumerate}
        \item To recover damages for lost profits, the plaintiff must show 
        with certainty that the breach caused the alleged amount of loss. 
        \emph{Kenford Co. v. Erie County}, \emph{Ashland Management Inc. v. 
        Janien}.
        \item The certainty rule has less punch today because we're better at 
        calculating damages.
        \item The \textbf{new business rule}: courts are wary of awarding 
        speculative damages to new ventures.
        \item Consistency in earlier performance can strengthen the case for 
        speculative damages.
        \item Statistical evidence of the performance of similar ventures can 
        help establish damages. \emph{Contemporary Mission, Inc. v. Famous 
        Music Corp.}
        \item \textbf{All or nothing rule}: one premise of \emph{Kenford} is 
        that there is some level of certainty above which the plaintiff can 
        fully recover, and below which the plaintiff can recover nothing. This 
        is wrong---the plaintiff should be compensated for the value of the 
        \emph{chance} to earn a profit, even if it is not certain that a 
        profit would result.
        \item Fuller and Eisenberg propose a formula for calculating damages 
        based on probability based on the Capital Asset Pricing Model: damages 
        should be awarded in proportion their likelihood.  So if a venture has 
        as 10\% chance of \$20 million and a 90\% chance of \$10 million, the 
        award should be (0.10 x \$20 million) + (0.90 x \$10 million), or \$11 
        million.
    \end{enumerate}
    \item \textbf{Liquidated damages}:
    \begin{enumerate}
        \item Liquidated damages clauses must reasonably approximate 
        anticipated or actual damages. \emph{Wasserman's Inc. v. Middletown}.
        \item Liquidates damages were traditionally frowned upon because they 
        could be punitive. But today, there's a resurgence, especially among 
        law and economics scholars, because the parties are in the best 
        position to calculate damages.
        \item Underliquidated damages---an amount less than actual estimated 
        damages---are usually enforced.
        \item Other techniques include limiting liability for consequential 
        damages or provisions that sellers' responsibilities are limited to 
        replacing or repairing defective goods.
    \end{enumerate}
\end{enumerate}

\subsubsection{Specific Performance}

\begin{enumerate}
    \item At common law, only equity courts could award specific performance.
    \item Courts will rarely award specific performance when money damages 
    will suffice. \emph{London Bucket Co. v. Stewart}. For efficiency reasons, 
    other courts are willing to award injunctions when damages would be 
    sufficient---e.g., to encourage negotiation. \emph{Walgreen Co. v. Sara 
    Creek Property Co.}
    \item Contracts cannot provide for injunctions because the court has sole 
    discretion in determining whether to issue an injunction. \emph{Stokes v. 
    Moore}.
    \item Specific performance is appropriate when the \textbf{value of property is 
    uncertain}, but not because the property is unique---because every piece 
    of property is unique.  \emph{Van Wagner Advertising Corp. v. S \& M 
    Enterprises}.
    \item % TODO UCC 2-709, 2-716
\end{enumerate}

\subsubsection{Reliance Damages}

\begin{enumerate}
    % TODO: add cases from consideration
    \item Reliance damages \textbf{compensate a party for costs incurred}.
    \item When there are \textbf{no measurable lost profits}, reliance damages 
    may be the only way to avoid injustice. \emph{Security Stove \& Mfg. Co.  
    v.  American Rys. Express Co.} Lost potential profits cannot be calculated 
    with certainty, but expenditures can---so when lost profits are uncertain, 
    reliance damages can be appropriate. \emph{Beefy Trail, Inc. v. Beefy King 
    Int'l, Inc.}
    \item Expenses made before the contract can be recovered under reliance 
    damages if the breach caused them to be \textbf{wasted}. \emph{Anglia 
    Television Co. v. Reed}.
    \item Reliance damages should be offset if the seller can prove that 
    the buyer's venture would not have been profitable enough to cover its 
    expenses. \emph{L. Albert \& Son v. Armstrong Rubber Co.}
    \item A promisee can recover reliance damages from a promisor after the 
    promise was made, but not for similar activity prior to the promise. For 
    instance, as part of a contract with B, C ships goods to A. After several 
    shipments, B defaults. A promises that it will pay C for the goods it 
    ships to A. C can recover reliance damages for the cost of shipments after 
    A's promise, but not for prior shipments. \emph{Westside Galvanizing 
    Servs., Inc. v. Georgia-Pacific Corp.}
\end{enumerate}

\subsubsection{Restitution Damages}

\begin{enumerate}
    \item The non-breaching party will sometimes be satisfied with rescission 
    of the contract and restitution of any value conferred on the breaching 
    party. The purpose of restitution is to \textbf{prevent unjust 
    enrichment}. \emph{Osteen v. Johnson}.
    \item Restitution damages are based on \textbf{benefit incurred}. while 
    reliance damages are based on \textbf{cost incurred}.
    \item \textbf{Unjust enrichment}: measured by the benefit the plaintiff 
    conferred on the defendant.
    \item \textbf{\emph{Quantum meruit}}: measured by the value of the 
    plaintiff's services. The plaintiff can recover even if performance of the 
    contract would have resulted in a loss. \emph{United States v. Algernon 
    Blair, Inc.}
    \item \textbf{Restitution vs. quantum meruit}---bank error in your favor:
    \begin{enumerate}
        \item Restitution: you have to give the money back. Otherwise, you'd 
        be unjustly enriched.
        \item Quantum meruit: the bank can recover nothing because it did not 
        render any service of value.
    \end{enumerate}
    \item Restitution damages are not available when parties have performed 
    all of the contract except payment. \emph{Oliver v. Campbell}.
    \item The promisee in an unprofitable contract can recover the value of 
    benefit incurred, even if it exceeds the contract price. \emph{United 
    States v. Algernon Blair, Inc.} For costs incurred (i.e., reliance 
    damages), the damages award should not exceed the contract price.
    \begin{enumerate}
        \item Restatement \S\ 349: reliance damages are an alternative 
        to expectation damages---but they should only apply when expectation 
        damages are too uncertain or should be limited for other reasons.
    \end{enumerate}
    \item At common law, the breaching party could not recover. However, the 
    modern trend is to \textbf{allow the breaching party to recover for benefit 
    conferred} (minus actual injuries). To allow the non-breaching party to 
    retain the benefit conferred beyond actual injury would be unjust 
    enrichment. \emph{Kutzin v. Pirnie}, Restatement \S\ 374(1).
    \begin{enumerate}
        \item ``~.~.~.~the breaching party is entitled to the reasonable value 
        of its services less any damages caused by the 
        breach.''\footnote{Casebook p. 362.}
        \item ``~.~.~.~a party injured by breach of contract is entitled to 
        retain nothing in excess of that sum which compensates him for the 
        loss of his bargain.''\footnote{Casebook p. 364.} \emph{Vines v. 
        Orchard Hills, Inc.}
    \end{enumerate}
\end{enumerate}

\subsection{Assent}

\subsubsection{Interpretation}

\begin{enumerate}
    \item \textbf{Subjectivity and objectivity}:
    \begin{enumerate}
        \item Contracts generally require a \textbf{``meeting of the 
        minds.''}\footnote{Casebook p. 368.}
        \item What happens in cases of misinterpretation? There's a tension 
        between subjective and objective intent.
        \item Courts will hold parties to a party's ``outward expression as 
        expressing his intent, rather than his ``secret and unexpressed 
        intention.''\footnote{Casebook p. 373.} \emph{Lucy v. Zehmer}.
        \item No contract is made when ``the whole transaction between the 
        parties was a frolic and a banter.''\footnote{Casebook p. 374.} 
        \emph{Keller v.  Holderman}.
        \item If there are \textbf{``latent ambiguities''} (e.g., multiple 
        ships named ``Peerless''), courts will accept additional evidence of 
        the parties' intent. If there was no meeting of the minds, there was 
        no contract. \emph{Raffles v. Wichelhaus}, \emph{Oswald v. Allen}.
        \begin{enumerate}
            \item ``If neither party can be assigned the greater blame for the 
            misunderstanding, there is no nonarbitrary basis for deciding 
            which party's understanding to enforce, so the parties are allowed 
            to abandon the contract without liability.''\footnote{Casebook p. 
            381.} \emph{Colfax Envelope Corp. v. Local No. 458-3M}.
        \end{enumerate}
        \item Courts consider many sources in determining the meaning of a 
        word, including dictionaries, trade usage, regulatory usage, and plain 
        meaning.  \emph{Frigaliment Importing Co. v. B.N.S. Intern. Sales Co.}
        \item Parties can communicate in codes unique to them as long as both 
        parties agree on the meaning. \emph{Falck v. Williams}.
        \item Contracts generally require a meeting of the minds, but intent 
        is irrelevant if the other party \textbf{could not reasonably know} 
        the other's intent (and if the objective meaning was clear). 
        \emph{Embry v.  Hargadine, McKittrick Dry Goods Co.}
        % TODO restatement 20, 201 -- and in main section
        % TODO berring, `MCC-Marble Ceramic Center'
        \item If one party knows the other's subjective intent, he can't fool 
        him by making an agreement that intentionally contradicts that intent 
        (e.g., if a buyer thinks he's buying unencumbered property, the seller 
        can't conceal the current tenant's purchase option). \emph{Mayol v. 
        Weiner Companies, Ltd.}
        \item \textbf{Classical vs. modern interpretation}:
        \begin{enumerate}
            \item Classical contract law largely \textbf{disregarded the 
            parties' intent}.
            \item There are \textbf{four principles of interpretation} in 
            modern contract law:\footnote{Casebook p. 394--95.}
            \begin{enumerate}
                \item The more reasonable meaning prevails. If both parties 
                attach different subjective meanings to an expression  and 
                they are not equally reasonable.
                \item But if the two meanings are equally reasonable, neither 
                prevails.
                \item If the parties attach the same meaning, that meaning 
                prevails even if it is unreasonable.
                \item If A and B attach different meanings, and A knows B's 
                meaning but B doesn't know A's, B's meaning prevails even if 
                it is less reasonable.
            \end{enumerate}
        \end{enumerate}
        \item ``~.~.~.~where the parties have not clearly expressed the 
        \textbf{duration of a contract}, the courts will imply that they 
        intended performance to continue for a reasonable 
        time.''\footnote{Casebook p.  401.}
        % TODO restatement 204 and in main sec
    \end{enumerate}
    \item \textbf{Problems of interpretation:}
    \begin{enumerate}
        \item Some courts hold that they should take \textbf{``material 
        circumstances''} into account to honor the parties' intent. 
        \emph{Spaulding v. Morse}. Others disagree. \emph{Lawson v. Martin 
        Timber Co.}
        \item Some amount of interpretation is always necessary---e.g., 
        ``fetch some soupmeat.''\footnote{Casebook p. 405.}
    \end{enumerate}
    \item \textbf{Trade usage}:
    \begin{enumerate}
        \item \textbf{Trade usage} applies to terms of agreements between 
        parties involved in a particular trade, e.g., textiles. \emph{Foxco 
        Industries, Ltd. v. Fabric World, Inc.}
        \begin{enumerate}
            % TODO UCC 2-202
            % TODO restatement 221, 222 and in main sec
            \item \textbf{Trade quantities} can vary if the usage is customary 
            within the trade---e.g., 4,000 shingles can really mean 2,500.  
            \emph{Hurst v.  W.J. Lake \& Co.}
            % TODO restatement 220, p. 410, and in main sec
            \item If one party is unaware of the trade usage, but should have 
            known, there is a fundamental difference in intent and thus no 
            contract. \emph{Flower City Painting Contractors, Inc. v. Gumina} 
            (following \emph{Raffles v. Wichelhaus}).
        \end{enumerate}
        % TODO ucc 1-201, 1-205, 2-208 and in main sec
    \end{enumerate}
\end{enumerate}

\subsubsection{Offer and Revocation}

\begin{enumerate}
    \item Key concepts: \textbf{intent, clarity, revocation}.
    \item \textbf{What constitutes an offer?}
    \begin{enumerate}
        % todo restatement 24, 26
        % TODO ucc 2-328; and see Hoffman v. Horton and the following note on 
        % auctions (reserve rules)
        \item \textbf{Negotiation vs. firm offer}: there is an important 
        distinction between intent to find out if the other party is 
        interested and intent to make a definite offer. \emph{Lonergan v. 
        Scolnick}.
        \item If a party reserves the right to not accept an offer, a proposal 
        is only an \textbf{invitation} to submit an offer. \emph{Regent 
        Lighting Corp. v. CMT Corp.}
        \item \textbf{Unilateral offer}: a binding obligation arises from 
        newspaper ads if ``the facts show that some performance was promised 
        in positive terms for something requested.''\footnote{Williston. 
        Casebook p.  418.} ``~.~.~.~where the offer is \textbf{clear, 
        definite, and explicit}, and leaves nothing open for negotiation, it 
        constitutes an offer, acceptance of which will complete the 
        contract.''\footnote{Casebook p.  419.} \emph{Lefkowitz v. Great 
        Minneapolis Surplus Store}.
        \begin{enumerate}
            \item But, courts have imposed limits---e.g., an ad offering 11\% 
            financing on a car does not constitute an offer because not 
            everyone will qualify for that rate. \emph{Ford Motor Credit Co. 
            v. Russell}.
            \item Is a newspaper ad an offer or an invitation to negotiate? 
            Courts generally hold that it is an offer if it \textbf{requires 
            the consumer to perform a specific act} (e.g., first come, first 
            served). But does that align with consumers' reasonable 
            expectations? Does traveling to the store count as an act? 
            \emph{Donovan v. RRL Corp.}
            \item Displaying something in a shop window with a price tag is an 
            invitation, not an offer. \emph{Fisher v. Bell}.
        \end{enumerate}
        \item People are not legally bound when they make appointments or 
        reservations.\footnote{Casebook p. 424.}
    \end{enumerate}
    \item \textbf{Lapse, rejection, and counteroffer}:
    \begin{enumerate}
        \item ``An offer may be \textbf{terminated} in a number of ways, as, 
        for example, where it is \textbf{rejected} by the offeree, or where it 
        is not accepted by him within the \textbf{time fixed}, or, if no time 
        is fixed, within a \textbf{reasonable time}.  An offer terminated in 
        either of these ways ceases to exist and cannot therefore be 
        accepted.''\footnote{Casebook p. 427.} \emph{Akers v. J.B. Sedberry, 
        Inc.}
        \item A \textbf{qualified acceptance}, subject to a condition, does 
        not create a contractual obligation if the other party does not 
        satisfy the condition---e.g., offering to buy a house only if the 
        previous owners left behind the furniture. \emph{Ardente v. Horan}.
        \begin{enumerate}
            \item If the conditional acceptance does not add any conditions to 
            the contract, but rather releases the offeror from an obligation, 
            it is an ordinary binding acceptance (not a conditional 
            acceptance)---e.g., buying land from a railroad but relieving the 
            railroad from its obligation to remove tracks from the land. 
            \emph{Rhode Island Dep't of Transp. v.  Providence \& Worcester 
            R.R.}
            \item A grumbling acceptance is still an acceptance. ``The 
            notation amounted to no more than saying I don't like your offer, 
            I don't think it's right or fair, but I accept it. That and 
            nothing more.''\footnote{Casebook p. 435.} \emph{Price v. Oklahoma 
            College of Osteopathic Medicine and Surgery}.
            \item The condition is not part of the offer if the offeree's 
            acceptance does not depend on the condition---e.g., renewing a 
            lease with a request to build a cookroom. \emph{Culton v. 
            Gilchrist}.
        \end{enumerate}
        \item Under classical contract law, the \textbf{mirror image rule} 
        held that no contract was formed if the acceptance differed from the 
        offer in any way. Modern contract law has softened the rule in two 
        ways:
        \begin{enumerate}
            \item UCC \S\ 2-207---see below (battle of the forms).
            \item Restatement Second \S\ 59: an acceptance containing 
            additional terms is binding if acceptance does not depend on the 
            offeror's assent to the additional terms.
        \end{enumerate}
        \item A seller can \textbf{renew an offer by implication}.  
        \emph{Livingstone v. Evans}.
        \begin{enumerate}
            \item For instance---S: I'll sell for \$1,800. B: How about 
            \$1,600?  S: Sorry, can't go any lower. B: Ok, then, \$1,800.
            \item S renewed the offer by implication.
        \end{enumerate}
        \item If the offeror \textbf{dies or becomes incapacitated}, and the 
        offeree is unaware, then classical contract law holds that the 
        offeror's estate is bound. Some modern courts have criticized the rule 
        as out of step with the parties' intent. Eisenberg proposes that the 
        offeree be able to recover reliance damages but not expectation 
        damages.\footnote{Casebook pp. 437--38.}
        \item \textbf{Receipt} happens when the person conducting the 
        transaction receives it or ``when it would have been brought to his 
        attention if the organization had exercised due 
        diligence~.~.~.''\footnote{Casebook p.  442.}
    \end{enumerate}
    \item \textbf{Revocation}:
    \begin{enumerate}
        \item Under classical contract law, the \textbf{unilateral contract 
        rule} held that an offer ``could be revoked at any time before the 
        designated act had been completed, even if performance of the act had 
        begun.''\footnote{Casebook p. 446.}
        \begin{enumerate}
            \item This rule frustrated the parties' expectations and defied 
            the interests of offerors as a class.\footnote{Casebook pp. 
            446--47.}
            \item So, the Restatement (First) drew a distinction between 
            \textbf{performing} and \textbf{preparing to perform}, which the 
            court followed in \emph{Ragosta v. Wilder}. But the distinction 
            can be hard to justify. For instance, say the unilateral offer is 
            that I'll give you \$1,000 to cross the Brooklyn Bridge. If you 
            take one step on the bridge, there is a contract. But if you spend 
            hours preparing, there is no contract.
        \end{enumerate}
        \item Common law: ``an offer is \textbf{freely revocable}, even if the 
        offeror has promised to hold it open, unless that promise is supported 
        by \textbf{consideration or reliance}.''\footnote{Casebook p. 453.} 
        See \emph{Ragosta v. Wilder}. But there have been recent changes:
        \begin{enumerate}
            \item UCC: merchants can make a ``firm offer'' (i.e., an 
            irrevocable offer) without the need for consideration. The offeror 
            must be a merchant, etc.\footnote{Casebook p. 453.}
            \item CISG Art. 16 allows an offeror to make an irrevocable offer 
            without these restrictions.
        \end{enumerate}
        \item \textbf{Promissory estoppel prevents a subcontractor from 
        revoking its offer} once the contractor has acted upon the 
        subcontractor's promise.  \emph{Drennan v. Star Paving Co.}
        \begin{enumerate}
            \item \textbf{Asymmetry}: subcontractors are bound to the general, 
            but the general is not bound to the subcontractor, creating 
            incentives for the general contractor to act unethically:
            \begin{enumerate}
                \item \emph{Bid shopping}: using the lowest bid to negotiate 
                lower bids from others.
                \item \emph{Bid chopping}: pressuring the subcontractor to 
                make a lower bid.
                \item \emph{Bid peddling}: a subcontractor waits until other 
                bids are in and then undercutting them, avoiding the cost of 
                estimating his own bid.
            \end{enumerate}
            \item Most courts have followed \emph{Drennan}, but at least one 
            has deviated.\footnote{Casebook p. 453.}
            \item \S\ 90 reliance damages may not be available when the 
            general contractor engages in these kinds of unethical practices.  
            \emph{Preload Technology, Inc. v. A.B. \& J. Construction Co., 
            Inc.}
        \end{enumerate}
        \item \textbf{Option contracts}:
        \begin{enumerate}
            \item An option contract is an offer in which the offeror promises 
            to keep the offer open for a certain period of time. For instance, 
            a seller grants a buyer the option to buy his house for \$1,000 
            anytime during the next month.
            \item ``An offer which the offeror should reasonably expect to 
            induce action or forbearance of a substantial character on the 
            part of the offence before acceptance and which does induce such 
            action or forbearance is binding as an option contract to the 
            extent necessary to avoid injustice.''
            \item The distinction between \S\S\ 45 (creating a binding option 
            contract when the offeree begins to perform) and 87 is that an 
            offeree who has begun performance can recover expectation damages, 
            while an offeree who has not begun performance can only recover 
            reliance damages.\footnote{Casebook p. 455.}
        \end{enumerate}
    \end{enumerate}
\end{enumerate}

\subsubsection{Modes of Acceptance}

\begin{enumerate}
    \item The offeror is the master of his offer. He can specify a mode of 
    acceptance.
    \item \textbf{Acceptance by act}:
    \begin{enumerate}
        \item A promise becomes binding when the offeree acts on the offeror's 
        request---e.g., if you promise to devise your property to someone if he 
        takes care of you, his acts of care count as acceptance. 
        \emph{Klockner v. Green}.
        \item ``So long as the outstanding offer was known to him, a person 
        may accept an offer for a unilateral contract by rendering 
        performance, even if he does so primarily for \textbf{reasons 
        unrelated to the offer}''\footnote{Casebook p. 466.}---e.g., going on 
        an ordinary fishing trip and catching Diamond Jim III. \emph{Simmons 
        v. United States}.
        \item Payment of rewards shouldn't be based on knowledge of the 
        reward. Do we want people to avoid doing their civic duty unless they 
        know they'll get paid? \emph{Stephens v. Memphis}.
        \item \textbf{Performing a condition counts as acceptance of the 
        offer.} For instance, if the Carbolic Smoke Ball manufacturers promise 
        a reward for anyone who uses the product and gets sick, anyone who 
        performs those conditions has accepted the offer and can recover the 
        reward. \emph{Carlill v. Carbolic Smoke Ball, Inc.}
    \end{enumerate}
    \item \textbf{Subjective acceptance}:
    \begin{enumerate}
        \item Notice of acceptance is distinct from acceptance itself. If 
        notice is not required, subjective acceptance counts as acceptance to 
        make the offer binding. \emph{International Filter Co. v. Conroe Gin, 
        Ince, \& Light Co.}
    \end{enumerate}
    \item \textbf{Acceptance by conduct}:
    \begin{enumerate}
        \item \textbf{Regular conduct} can count as tacit assent to 
        contractual terms. ``~.~.~.~when an offeree accepts the offeror's 
        services without expressing any objections to the offer's essential 
        terms, the offeree has manifested assent to those 
        terms.''\footnote{Casebook p. 478.} \emph{Polaroid Corp. v. Rollins 
        Environmental Services (NJ), Inc.}
    \end{enumerate}
    \item \textbf{The effect of using a subcontractor's bid}:
    \begin{enumerate}
        \item If a general contractor uses a subcontractor's bid in submitting 
        its own bid, it is not bound to use the subcontractor's services. 
        There are several justifications for ``unequal treatment of generals 
        and subcontractors~.~.~.~''\footnote{Casebook p. 483.} \emph{Holman 
        Erection Co. v. Orville E. Madsen \& Sons, Inc.}
        \begin{enumerate}
            \item The statute underlying \emph{Holman} was later amended to 
            prevent general contractors from substituting subcontractors 
            unless the subcontractor was unable or unwilling to 
            perform.\footnote{Casebook p.  485.}
        \end{enumerate}
    \end{enumerate}
    \item \textbf{Silence as acceptance}:
    \begin{enumerate}
        \item By default, \textbf{silence does not constitute acceptance}. 
        \emph{Vogt v. Madden}.
        \item Restatement Second \S\ 69 recognizes \textbf{two exceptions}: 
        (1) when the offeree \textbf{silently takes offered benefits} and (2) 
        ``where one party \textbf{relies} on the other party's manifestation 
        of intention that silence may operate as 
        acceptance.''\footnote{Casebook p. 493.}
        \item ``Where the offeree with reasonable opportunity to reject 
        offered services \textbf{takes the benefit} of them under 
        circumstances which would indicate to a reasonable person that they 
        were offered with the expectation of compensation, he assents to the 
        term proposed and thus accepts the offer.''\footnote{Casebook p. 495.} 
        \emph{Laurel Race Courses v. Regal Const. Co.}
        \item Delay in rejection can count as acceptance. ``It will not do to 
        say that a seller of goods like these could wait indefinitely to 
        decide whether or not he will accept the offer of the proposed 
        buyer.''\footnote{Casebook p. 497.} \emph{Cole-McIntyre-Norfleet Co. 
        v. Holloway}. In some cases, there is a duty to promptly reply, and 
        unreasonable delay will count as acceptance---e.g., a hail insurer 
        waiting two months to send a rejection. \emph{Kukusa v. Home Mut. 
        Hail-Tornado Ins. Co.}
        \item Repeated orders for the same product can count as a 
        \textbf{standing offer}. If the buyer does not reject future 
        shipments, his silence counts as acceptance. \emph{Hobbs v. Massasoit 
        Whip Co.}
        \item Taking physical control of shipped goods counts as acceptance of 
        the shipment. \emph{Louisville Tin \& Stove Co. v. Lay}.
        \item If you enjoy the benefit of an unwanted thing, you have to pay 
        for it---e.g., an expired newspaper subscription. \emph{Austin v. 
        Burge}.
        \item A \textbf{negative-option} plan involves a subscription for 
        merchandise, like a book or record club. They differ from unordered 
        goods in that the customer contracts in advance. Silence indicates 
        continued acceptance.
    \end{enumerate}
    \item \textbf{Acceptance by electronic agent}:
    \begin{enumerate}
        \item VETA and E-Sign have clarified that electronic records and 
        signatures satisfy the statute of frauds.
        \item Contracts can be formed between a person and a computer, or even 
        between two computers.\footnote{Casebook p. 502.} ``~.~.~.~it is 
        conceivable that, with the useful life of this Act, electronic agents 
        may be created with the ability to act autonomously, and not just 
        automatically.''\footnote{Casebook p. 503.}
    \end{enumerate}
\end{enumerate}

\subsubsection{Implied-in-Law and Implied-in-Fact Contracts}

\begin{enumerate}
    \item \textbf{Implied-in-fact} contracts are true contracts in which 
    assent is implied, not explicit---e.g., raising your hand to bid at an 
    auction.\footnote{Casebook p. 508.}
    \item \textbf{Implied-in-law} contracts aren't really contracts. There's 
    no offer, acceptance, or assent. But to prevent unjust enrichment, someone 
    should be rewarded.
    \item ``Contracts implied in law, or as they are more commonly called 
    `quasi contracts,' are obligations imposed by law on grounds of justice 
    and equity. Their purpose is to \textbf{prevent unjust enrichment}. Unlike 
    express contracts or contracts implied in fact, quasi contracts do not 
    rest upon the assent of the contracting parties.''\footnote{Casebook p.  
    505.}
    \begin{enumerate}
        \item The \textbf{officious intermeddler} doctrine prevents foisting 
        labor upon another without consent. But the \textbf{emergency aid 
        exception} allows recovery if the intervener ``acted unofficiously and 
        with intent to charge''\footnote{Casebook p. 505.}---for instance, 
        medically necessary nursing services that the patient did not consent 
        to. \emph{Nursing Care Services, Inc. v. Dobos}
    \end{enumerate}
    \item Implied-in-fact contracts: the \emph{parties} decide that there is 
    a contract. Implied-in-law: the \emph{court} decides.
    \item When someone else confers value on you, and you know he expects 
    payment, your silence is assent to pay (i.e., the court will find a 
    contract implied in law---otherwise you'd be unjustly enriched). \emph{Day 
    v. Caton}.
    \item Damages vary:\footnote{Casebook p. 515.}
    \begin{enumerate}
        \item Implied-in-fact contracts are based on the parties' intentions, 
        so the proper remedy is compensatory damages.
        \item Implied-in-law contracts are based on unjust enrichment, so the 
        proper remedy is the value of the benefit acquired.
        \item Damages in an unjust enrichment (i.e., implied in law) claim are 
        measured by the benefit conferred. Damages in a quantum meruit claim 
        are measured by the reasonable value of the plaintiff's 
        services.\footnote{Casebook pp. 515--16.}
    \end{enumerate}
    \item \textbf{Employee handbooks}:
    \begin{enumerate}
        \item Provisions in employee handbooks are enforceable. \emph{Pine 
        River State Bank v. Mettille}.
        \item What happens if the employer modifies the handbook and the 
        employee continues to work? Does the employee's continued work 
        constitute acceptance of the modification? Courts are 
        split.\footnote{Casebook pp.  531--33.} Courts are also split on 
        whether to enforce disclaimers.\footnote{Casebook pp. 533--34.}
    \end{enumerate}
\end{enumerate}

\subsubsection{Preliminary Negotiations, Indefiniteness, and the Duty to 
Bargain in Good Faith}

\begin{enumerate}
    \item Agreements can be \textbf{too indefinite} to allow courts to fashion 
    remedies for breach, and are therefore unenforceable.
    \begin{enumerate}
        \item There is no contract if the parties lack a mutual understanding 
        of the essential parts of the agreement. The court cannot substitute 
        the missing terms if they are central to the contract. \emph{Academy 
        Chicago Publishers v. Cheever}. 
        \item ``An agreement to enter into an agreement upon terms to be 
        afterwards settled between the parties is a contradiction in 
        terms.''\footnote{Casebook p. 541.} \emph{Ridgway v. Wharton}.
        \item If the essential parts of an agreement are included and agreed 
        upon, the contract is enforceable even if other clauses are 
        omitted.\footnote{Casebook p. 541.} \emph{Berg Agency v. 
        Sleepworld-Willingboro, Inc.} Courts should \textbf{fill gaps} where 
        the parties' reasonable expectations are clear, but they should not 
        impose performance to which the party would not have agreed. 
        \emph{Rego v. Decker}.
        \item Parties may want to create incomplete contracts to allow future 
        flexibility. \emph{AROK Construction Co. v. Indian Construction 
        Services}.
        \item There is a general understanding in the trade that 
        subcontractors will enter into an agreement with the general 
        contractor after submitting a winning bid, even if it contained no 
        terms other than the price. If this were not the case, general 
        contractors would be unable to recover \S\ 90 reliance damages. 
        \emph{Saliba-Kringlen Corp. v. Allen Engineering Co.}
        \item The UCC ``gap-filler'' provisions fill the gaps that parties may 
        leave in contracts for sale of goods.
        \begin{enumerate}
            \item \textbf{Default rules}: background rules that the law reads 
            into a contract only if the parties left them out.
            \item \textbf{Mandatory rules}: not waivable by the parties, e.g., 
            the requirement of consideration. If the parties leave them out, 
            there is no contract.
        \end{enumerate}
        \item If there is no method in a lease for calculating rent, the court 
        cannot determine that the parties intended to be bound by fair market 
        value. \emph{Joseph Martin, Jr., Delicatessen, Inc. v. Schumacher}. 
        But some courts, applying a minority rule, find that the parties at 
        least would have intended rent to be ``reasonable.'' \emph{Moolenaar 
        v. Co-Build Companies, Inc.}
    \end{enumerate}
    \item \textbf{Offer and acceptance} vs. \textbf{preliminary 
    negotiations}:
    \begin{enumerate}
        \item Classical contract law drew a strict binary distinction between 
        offer and acceptance, which created a binding agreement, and 
        preliminary negotiations, which did not.\footnote{Casebook p. 536.}
    \end{enumerate}
    \item What about agreements to create \textbf{future agreements}? Again 
    classical contract law followed a strict binary: on the one hand, if the 
    future agreement was meant only to confirm the original agreement, the 
    original agreement was enforceable; on the other hand, if the parties 
    intended not to be bound unless they executed the future agreement, the 
    original agreement was not enforceable.
    \item Classical contract law recognized no duty to negotiate in good 
    faith,\footnote{Casebook p. 537.} though the parties themselves could 
    agree to good faith negotiations.\footnote{Casebook p. 553.} \emph{Channel 
    Home Centers v. Grossman}.
\end{enumerate}

\subsubsection{Parol Evidence}

\begin{enumerate}
    \item Thayer: some want a ``lawyer's Paradise'' where written instruments 
    have a precise, fixed meaning. But in reality, context is a ``fatal 
    necessity.''\footnote{Casebook p. 590.} 
    \item Should we allow oral evidence or require the parties to put their 
    entire agreement in formal, written terms? Williston would exclude oral 
    evidence unless similarly situated parties would have supplemented the 
    written agreement with oral terms. Corbin would allow evidence of intent 
    beyond the writing itself. Corbin has largely won.\footnote{Casebook p. 
    592. On Corbin's greater influence, see \emph{Interform Co. v. Mitchell 
    Constr. Co.}, p. 594.}
    \item The classical, Willistonian view: courts should \textbf{presume 
    complete integration}, ``and should admit evidence of consistent 
    additional terms only if there is substantial evidence that the parties 
    did not intend the writing to embody the entire 
    agreement.''\footnote{Casebook p. 594.} \emph{Hatley v. Stafford}.
    \item Restatement Second \S\ 209 determines whether an agreement is 
    integrated as ``a question of fact to be determined in accordance with all 
    relevant evidence.'' The Restatement follows Corbin, who argued that on 
    this issue, ``no relevant testimony should be 
    excluded.''\footnote{Casebook p. 595.}
    \item If the parties might reasonably have left terms out of the 
    agreement, the court can admit evidence of the parties' intent and oral 
    agreements. For instance, in a purchase option agreement, the parties 
    might not have written down a provision that the option was personal to 
    that party and nonassignable. \emph{Masterson v. Sine}.
    % TODO restatement 209, 210, 213--16, 228, 237, 239, 240
    % ucc 2-202
    \item UCC \S\ 2-202 reflects Corbin's influence in its focus on the intent 
    of the parties, rather than the practices of reasonable people. 
    \emph{Interform v. Mitchell Constr. Co.}
    \item UCC \S\ 2-202 provides that parol evidence may explain or supplement 
    a written agreement agreement, but not contradict it. According to the 
    court, a term is inconsistent if it contradicts or negates a term of the 
    writing. If it does not, it is admissible. \emph{Hunt Foods and 
    Industries, Inc. v. Doliner}.
    \begin{enumerate}
        \item The \emph{Hunt Foods} view of consistency is too narrow. A 
        better definition is ``the absence of reasonable harmony in terms of 
        the language and respective obligations of the 
        parties.''\footnote{Casebook p. 601.} \emph{Alaska Northern 
        Development, Inc. v. Alyeska Pipeline Service Co.}
    \end{enumerate}
    \item \textbf{Merger (or integration) clauses} provide that the written 
    contract is the entire agreement between the parties.\footnote{Casebook p. 
    602.}
    \begin{enumerate}
        \item Under classical contract law, courts presume complete 
        integration.
        \item Today, a merger clause is often not enough to prove complete 
        integration. Courts often hold that the parties must have actually 
        assented to integration.
        \item Courts must consider surrounding circumstances to determine 
        whether the parties actually assented to the merger clause. \emph{ARB 
        (American Research Bureau), Inc. v. E-Systems, Inc.}
        \item UCC \S\ 2-202 requires intent to integrate in addition to a 
        merger clause. In a written contract, the merger clause must be 
        conspicuous. \emph{Siebel v. Layne \& Bowler, Inc.}
    \end{enumerate}
    \item \textbf{Fraud exception}: otherwise inadmissible parol evidence is 
    admissible if it shows an ``invalidating cause'' of the written 
    agreement---e.g., lack of consideration, duress, mistake, illegality, or 
    fraud.\footnote{Casebook p. 604.}
    \item \textbf{Condition-to-legal-effectiveness exception}: 
    \begin{enumerate}
        \item The parol evidence rule does not apply when the occurrence or 
        nonoccurence of an event, by spoken agreement, is a condition to 
        making the written agreement binding or effective.\footnote{Casebook 
        p. 606.}
        \item For instance, A and B make a written business agreement, and 
        agree orally that the agreement will be null if the parties fail to 
        raise \$600,000 within 20 days. Evidence of the oral agreement is 
        admissible.
    \end{enumerate}
    % ucc 2-209(2), (4), (5)
    \item The parol evidence rule applies only to oral agreements made 
    \textbf{before or contemporaneously with a written, integrated contract}. 
    The rule doesn't apply to a later agreement that modifies the 
    integration.\footnote{Casebook p. 208.}
    \item Written contracts often provide that they can be modified only in 
    writing---a private Statute of Frauds. These are \textbf{no oral 
    modification (n.o.m.)} clauses.
    \begin{enumerate}
        \item Common law: oral modifications are enforceable notwithstanding 
        n.o.m. clauses, because the later oral agreement by implication modifies 
        the earlier written agreement containing the n.o.m. clause.
        \item Example: ``I'll pay you \$100 to lock me in this room. Don't let me 
        out for 10 hours, no matter what I say.'' You sign the agreement. At 
        common law, cries of ``I changed my mind, I'll pay you \$500 to let me 
        out!'' would modify the original agreement.
        \item UCC \S\ 2-209 makes two key changes to the common law:
        \begin{enumerate}
            \item \S\ 2-209(1): a modification needs no consideration to be 
            enforceable.
            \item \S\ 2-209(2): if a contract for the sale of goods contains a 
            n.o.m. clause, the modification must be in writing.
        \end{enumerate}
        \item But, \S\ 2-209(4) allows that an attempt at modification under \S\ 
        2-209(2) can operate as a waiver. But then \S\ 2-209(5) provides that a 
        party who has created a waiver for an executory (unperformed) part of the 
        contract can retract it if the other party has relied on the waiver.
    \end{enumerate}
\end{enumerate}

\subsubsection{Plain Meaning}

\begin{enumerate}
    \item ``~.~.~.~where language is clear and unambiguous, the focus of 
    interpretation is upon the terms of the agreement as \emph{manifestly} 
    expressed, rather than as, perhaps, silently intended.''\footnote{Casebook 
    p. 608.} \emph{Steuart v. McChesney}.
    \item Some courts hold that the judge's linguistic reference point is 
    necessarily different than the parties'. Parties should be allowed to 
    introduce reasonable alternative interpretations of the written agreement. 
    \emph{Mellon Bank, N.A. v. Aetna Business Credit, Inc.}.
    \item UCC \S\ 2-202 assumes that the written contract is not completely 
    integrated. Under the UCC, the court should admit parole evidence by 
    default unless the judge thinks the written agreement is unambiguous.
    \item Traynor vs. Kozinski:
    \begin{enumerate}
        \item Language lacks fixed meaning. ``Accordingly, rational 
        interpretation requires at least a preliminary consideration of all 
        credible evidence offered to prove the intention of the 
        parties.''\footnote{Casebook p. 617.} Traynor in \emph{Pacific Gas \& 
        Electric Co. v. G.W. Thomas Drayage \& Rigging Co.}
        \item There are some cases where the plain meaning of language really 
        is unambiguous. Kozinski grudgingly agreed that the external evidence 
        must be allowed under the California Supreme Court's holding in 
        \emph{Pacific Gas}---which ``we have no difficulty 
        understanding~.~.~.~, even without extrinsic evidence to guide 
        us.''\footnote{Casebook p. 621.} Kozinski in \emph{Trident Center v. 
        Connecticut General Life Ins. Co.}
        \item What exactly wrong with following the \emph{PG\&E} 
        rule? If the language is in fact unambiguous, the case will easily 
        fall at trial. The only cost is judicial inefficiency, which is costly 
        in terms of time and money, but not analytically.
    \end{enumerate}
\end{enumerate} 

\subsection{Form Contracts}

\begin{enumerate}
    \item \textbf{Battle of the forms and UCC \S\ 2-207}:
    \begin{enumerate}
        \item \textbf{Battle of the forms}: buyers and sellers exchange 
        elaborate preprinted form contracts. It would be prohibitively costly 
        for buyers to actually read the contracts---\textbf{rational 
        ignorance}.\footnote{Casebook pp. 639--40.}
        \item  The \textbf{last shot} rule: under classical contract law, 
        conduct can constitute acceptance---e.g., the buyer could tacitly 
        accept the terms of the deal by accepting and keeping the goods. In 
        that case, the last form the parties exchanged would 
        control.\footnote{Casebook pp.  641--42.}
        \item UCC \S\ 2-204: general formation of a contract.
        \item UCC \S\ 2-207: there can be a contract even if the language of 
        the forms differs. The conflicting language is removed and, if 
        necessary, replaced with gap-fillers. It rejects the mirror image 
        rule.
        \begin{enumerate}
            \item The premise behind \S\ 2-207 is that no rational person 
            reads the boilerplate terms.
        \end{enumerate}
        \item \S\ 2-207 does not apply to \textbf{non-form, fully negotiated} 
        agreements. \enquote{Under such circumstances, when the parties fully 
        negotiate each provision of a contract, a contract may be 
        \enquote{beyond the reach of 2-207 and adrift on the murky sea of 
        common law.}}\footnote{Casebook p. 644.} \emph{Columbia Hyundai, Inc. 
        v. Carll Hyundai, Inc.}
        \item When does a response become \textbf{``expressly conditional,''} 
        making it a counteroffer for purposes of \S\ 2-207? If the dickered 
        terms are identical but the undickered terms vary, the transaction 
        does not become ``expressly conditional'' and the response does not 
        become a counteroffer. \emph{Gardner Zemke Co. v. Dunham Bush, Inc.}
        \begin{enumerate}
            \item Courts are split on the meaning of ``expressly made 
            conditional.'' Some hold that that if acceptance is ``expressly 
            made conditional,'' it must be ``directly and distinctly 
            stated.''\footnote{Casebook p. 655.} Others apply looser 
            standards, e.g., requiring certain key phrases, or requiring the 
            offeree to demonstrate unwillingness to proceed unless he agrees 
            to the additional terms.
        \end{enumerate}
        \item \textbf{Knockout rule}: under 2-207, conflicting terms cancel 
        each other out.
        \item \textbf{``Materially alter''} under 2-207(2)(b): a term added in 
        an acceptance does not become part of the contract if it would 
        ``materially alter'' the contract. The UCC defines ``materially 
        alter'' to mean ``would result in [unreasonable] surprise or 
        hardship.'' So, for instance, a mandatory arbitration clause is not a 
        material alteration.\footnote{Casebook pp. 653--55.}
    \end{enumerate}
    \item \textbf{Rolling contracts}:
    \begin{enumerate}
        \item \textbf{``Money now, terms later''}: shrinkwrap agreements, in 
        which a buyer purchases software and must agree to additional license 
        terms before using it, are valid. \emph{ProCD, Inc. v.  Zeidenberg}. 
        The rule applies to both software and hardware.  \emph{Hill v. Gateway 
        2000}.
        \item Clickwrap agreements, in which a user must agree to terms before 
        using software, are valid, but they require ``[r]easonably conspicuous 
        notice of the existence of contract terms and unambiguous 
        manifestation of assent to those terms by 
        consumers~.~.~.''\footnote{Casebook p. 686.} \emph{Specht v. Netscape 
        Communications Corp.}
    \end{enumerate}
    \item \textbf{Interpretation and unconscionability in form contracts}:
    \begin{enumerate}
        \item Reformation is appropriate only if there is mutual mistake. 
        \emph{Sardo v. Fidelity \& Deposit Co.}
        \item Form contracts can be unenforceable on the basis of public 
        policy. \emph{Weaver v. American Oil Co.}
    \end{enumerate}
    \item Llewellyn on form contracts: 
    \begin{enumerate}
        \item Form contracts are attractive. But they can take on ``a massive 
        and almost terrifying jug-handled character.''\footnote{Casebook p. 
        694.} Nobody really reads them.
        \item So, nobody really assents to boilerplate terms. Rather, people 
        only really assent to the dickered terms.
        \item ``[A]ny contract with boiler-plate results in \emph{two} several 
        contracts: the \emph{dickered} deal, and the collateral one of 
        \emph{supplementary} boiler-plate.''\footnote{Casebook p. 696.}
    \end{enumerate}
\end{enumerate}

\subsection{Mistake and Unexpected Circumstances}
\begin{enumerate}
    \item \textbf{Unilateral mistake (mechanical error)}: courts will grant 
    relief if (1) the other party has not relied on the error or (2) reliance 
    damages will not compensate the other party. See the Nolan Ryan baseball 
    card case.
    \begin{enumerate}
        \item % TODO restatement 154 (+153?)
    \end{enumerate}
    \item \textbf{Mistakes in transcription}: if there is proof of an 
    agreement, courts will reform an error in transcription as long as ``there 
    has been no prejudicial change of position by the other party while 
    ignorant of the mistake.'' \emph{Travelers Ins. Co. v. Bailey}. But if the 
    parties purposely based the contract on an uncertain event, there will be 
    no reformation. \emph{Chimart Associates v. Paul}.
    \item \textbf{Tacit assumptions} are present in every contract---e.g., the 
    sun will rise tomorrow.
    \item \textbf{Mutual mistake}: tacit or explicit assumptions in the 
    agreement turn out to be incorrect.
    \begin{enumerate}
        \item The \emph{Sherwood} rule held that a mistake excuses performance 
        when it relates to the ``substance'' or ``root'' of the contract, but 
        not if it relates to something peripheral. \emph{Sherwood v. Walker}. 
        But the \emph{Sherwood} rule was all but overruled shortly after. 
        \emph{Nester v. Michigan Land \& Iron Co.}
        \item \textbf{Impossibility}: when the parties' mistke renders 
        performance impossible, performance is excused. \emph{Griffith v. 
        Brymer}.
        \item If there was no fraud and neither party could have known about 
        the mistake, the parties cannot rescind the agreement. \emph{Wood v. 
        Boynton.}
        \item When the market price of an item depends on collective judgment 
        (e.g., fine art), there is no mistake if the collective judgment later 
        shifts (e.g., the art world discovers that a different artist painted 
        a piece). \emph{Firestone \& Parson, Inc. v. Union League of 
        Philadelphia}.
        \item ``~.~.~.~rescission is indicated when the mistaken belief 
        relates to a \textbf{basic assumption} of the parties upon which the 
        contract is made, and which materially affects the agreed performance 
        of the party.'' \emph{Lewanee County Board of Health v. Messerly}.
        \item The buyer's assumption of risk (e.g., through an as-is) clause 
        can enforce performance despite a mistake. But, some courts have held 
        that the buyer's assumption of risk os only relevant when the parties 
        were \emph{aware} of the possibility that they were wrong. 
        \emph{Beachcomber Coins, Inc. v. Boskett.}
    \end{enumerate}
    \item \textbf{Unexpected circumstances}:
    \begin{enumerate}
        \item \textbf{Implied conditions}: when performance depends on the 
        continued existence of a person or thing, that person or thing is an 
        implied condition. When that person or thing perishes, performance is 
        excused. \emph{Taylor v. Caldwell} (the plaintiff could rescind the 
        contract because a concert hall burned down).
        \item Contracts involve many \textbf{tacit assumptions}.  \enquote{We 
        `just know' that the burning of a music hall violates a tacit 
        assumption of the parties who executed a contract for hiring it for a 
        few days; we `just know' that a two per cent increase in the price of 
        beans does not violate a tacit assumption underlying a contract to 
        deliver a ton of beans for a fixed price.}\footnote{Casebook p. 769.}
        \item \textbf{Impracticability}: performance becomes ``impossible in 
        legal contemplation'' if it is beyond practicality---e.g., retrieving 
        underwater gravel. \emph{Mineral Park Land Co. v. Howard}. But 
        gambling on your own abilities and failing does not count as 
        impracticability---e.g., promising to deliver an engineering 
        breakthrough. \emph{United States v. Wegematic Corp.} Bargaining on 
        uncertainty likewise does not invalidate a contract if uncertain term 
        turns out to be unfavorable. \emph{Missouri Public Service Co. v. 
        Peabody Coal Co.}
    \end{enumerate}
    \item \textbf{The duty to perform in good faith}:
    \begin{enumerate}
        % TODO rest 205, 237
        \item ``~.~.~.~in every contract there exists an \textbf{implied 
        covenant of good faith and fair dealing}.''\footnote{Casebook p. 890.} 
        \emph{Kirke La Shelle Co. v. Paul Armstrong Co.}
        \item Parties implicitly agree to not interfere with the other's 
        performance. \emph{Patterson v. Meyerhofer}.
        \item But, ``[m]ere difficulty of performance will not excuse a breach 
        of contract''---even if one of the parties contributed to that 
        difficulty by affecting the market. It's ok to interfere with 
        performance indirectly. \emph{Iron Trade Products Co. v. Wilkoff Co.}
        \item Good faith in the UCC means ``honesty in fact and the observance 
        of reasonable commercial standards of fair 
        dealing.''\footnote{Casebook p. 891.} Both the UCC and the Restatement 
        (Second) require good faith performance.
        \item What does ``good faith'' mean?\footnote{Casebook pp. 892--93.}
        \begin{enumerate}
            \item Farnsworth: it's a fundamental idea of contract law. It implies 
            terms in the contract.
            \item Summers: the ``excluder'' analysis asks, what does a judge want 
            to rule out by use of the phrase ``good faith''? It has no meaning on 
            its own, but it serves to exclude many heterogeneous forms of bad 
            faith.
            \item Burton: the ``forgone opportunity analysis'' argues that ``good 
            faith'' limits what a party can do in performance, so bad faith is to 
            ``recapture opportunities forgone upon 
            contracting~.~.~.''
            \item Courts apply all three standards.
        \end{enumerate}
    \end{enumerate}
\end{enumerate}

\subsection{Secondary Sources}

% TODO: add restatament and UCC sections throughout

\subsubsection{UCC}

\begin{enumerate}
    \item % TODO
    % TODO add to overview 1-201(20)
    % TODO add to overview 2-202
    % TODO add to overview 2-204
    % TODO add to overview 2-205
    % TODO add to overview 2-206
    % TODO add to overview 2-207
    % TODO add to overview 2-208
    % TODO add to overview 2-209
    % TODO add to overview 2-302
    % TODO add to overview 2-328
    % TODO add to overview 2-712
    % TODO add to overview 2-713
    % TODO add to overview 2-714
    % TODO add to overview 2-715
    % TODO add to overview 2-716
    % TODO add to overview 2-718
\end{enumerate}

\subsubsection{Restatement (Second))}

\begin{enumerate}
    \item % TODO
\end{enumerate}
